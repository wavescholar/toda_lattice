% !TEX TS-program = xelatex
\documentclass[12pt, letterpaper]{article}

% ---------- Shared document settings ----------
\usepackage[margin=1in]{geometry}
\usepackage{amsmath, amssymb, amsthm}
\usepackage{graphicx}
\usepackage{xcolor}
\usepackage{hyperref}

% ---------- Additional utilities ----------
\usepackage{microtype}
\usepackage{mathtools}
\usepackage{bbm}
\usepackage{enumitem}
\usepackage{booktabs}
\usepackage[nameinlink]{cleveref}

\hypersetup{
  colorlinks=true,
  linkcolor=blue!55!black,
  citecolor=blue!55!black,
  urlcolor=blue!55!black,
  pdfauthor={},
  pdftitle={Special Topics: Short Notes},
  pdfcreator={}
}

% ---------- Theorems & macros ----------
\theoremstyle{plain}
\newtheorem{theorem}{Theorem}[section]
\newtheorem{lemma}[theorem]{Lemma}
\newtheorem{corollary}[theorem]{Corollary}

\theoremstyle{definition}
\newtheorem{definition}[theorem]{Definition}
\newtheorem{example}[theorem]{Example}
\newtheorem{remark}[theorem]{Remark}

\newcommand{\R}{\mathbb{R}}
\newcommand{\E}{\mathbb{E}}
\newcommand{\Prob}{\mathbb{P}}
\newcommand{\Var}{\operatorname{Var}}
\newcommand{\Cov}{\operatorname{Cov}}
\newcommand{\1}{\mathbbm{1}}

% ---------- Title ----------
\title{Special Topics: \\ Short Notes}
\author{}
\date{\today}

\begin{document}
\maketitle

\begin{abstract}
Short notes in topics from HDP and Analysis 
\end{abstract}

\section{Bernstein Theorem}

% --- Bernstein's inequality on the unit circle and the Erdős--Lax refinement ---

\begin{theorem}[Bernstein's inequality on $\mathbb T$]
Let $p$ be a complex polynomial of degree $n$. Then
\[
\max_{|z|=1}|p'(z)| \;\le\; n\,\max_{|z|=1}|p(z)|.
\]
Moreover, the constant $n$ is sharp.
\end{theorem}

\begin{proof}
Write $p(z)=z^{n}F(z)$, where $F$ is analytic in the exterior disk
$\{z:|z|>0\}$ and holomorphic at $\infty$ with $F(\infty)=\text{(leading coefficient)}$.
For $|z|=1$ we have
\[
p'(z)=n z^{n-1}F(z)+z^{n}F'(z).
\]
Set $M:=\max_{|w|=1}|p(w)|=\max_{|w|=1}|F(w)|$. Since $F$ is analytic on
$\{|w|\ge 1\}\cup\{\infty\}$ and bounded on $|w|=1$, the maximum-modulus
principle on the exterior domain implies $\sup_{|w|\ge 1}|F(w)|=M$.
Fix $z$ with $|z|=1$ and apply Cauchy’s estimate to $F'$ on the circle
$|w|=R>1$:
\[
|F'(z)| \;\le\; \frac{1}{R-1}\,\max_{|w|=R}|F(w)| \;\le\; \frac{M}{R-1}.
\]
Letting $R\to\infty$ gives $F'(z)=0$ for $|z|=1$, hence
$|p'(z)|=n|F(z)|\le nM$. Taking the maximum over $|z|=1$ yields the claim.
\end{proof}

\begin{remark}[Sharpness]
Equality holds for $p(z)=c\,z^{n}$ (any $c\in\mathbb C$), since
$\max_{|z|=1}|p'(z)|=n|c|=\;n\,\max_{|z|=1}|p(z)|$.
More generally, equality occurs for rotations of such extremals.
\end{remark}

\begin{theorem}[Erdős--Lax refinement]
If $p$ has no zeros in the open unit disk $\{z:|z|<1\}$, then
\[
\max_{|z|=1}|p'(z)| \;\le\; \frac{n}{2}\,\max_{|z|=1}|p(z)|,
\]
and the factor $\tfrac{1}{2}$ is best possible.
\end{theorem}

\begin{proof}[Proof sketch]
If $p$ is zero-free in $|z|<1$, then $F(z):=\dfrac{z\,p'(z)}{p(z)}$ is analytic on
$|z|<1$. A standard Herglotz–Carathéodory argument applied to
$G(z):=1-\dfrac{z\,p'(z)}{n\,p(z)}$ (which has nonnegative real part on $|z|<1$)
gives $|z\,p'(z)|\le \tfrac{n}{2}|p(z)|$ for $|z|=1$.
Sharpness is witnessed by extremals of the form
$p(z)=c\,(z-a)^{n}$ with $|a|>1$ (limit cases as $|a|\downarrow 1$).
\end{proof}

\begin{remark}[{Comparison with Markov on $[-1,1]$}]
On the interval $[-1,1]$ one has Markov's inequality
$\max_{[-1,1]}|q'|\le n^{2}\max_{[-1,1]}|q|$ for algebraic polynomials $q$,
illustrating the stronger $n^{2}$ growth on an interval versus the $n$ (or $n/2$)
growth on the unit circle.
\end{remark}


\section{Bernstein on the circle, Erdős--Lax, and contrasts with Markov on \([-1,1]\)}
\label{sec:bernstein-markov}

\subsection{Context and scope}
This note distills our discussion of Bernstein-type inequalities for complex
polynomials on the unit circle, the Erdős--Lax refinement in the zero-free
case, and how these compare with the classical Markov inequality on the real
interval \([-1,1]\). We also record the (separate) Bernstein polynomials used
in the constructive proof of the Weierstrass approximation theorem and explain
how the various ``Bernstein'' results are related in spirit but address
different questions.

\subsection{Bernstein's inequality on the unit circle}
\begin{theorem}[Bernstein inequality on $\mathbb T$]
\label{thm:bernstein}
Let $p$ be a complex polynomial of degree $n$. Then
\[
\max_{|z|=1}|p'(z)| \;\le\; n\,\max_{|z|=1}|p(z)|,
\]
and the constant $n$ is sharp.
\end{theorem}

\begin{proof}[Proof sketch (Cauchy/maximum-modulus)]
Fix $R>1$ and apply Cauchy's integral formula for derivatives to $p$ on the
circle $|w|=R$:
\[
p'(z)=\frac{1}{2\pi i}\int_{|w|=R}\frac{p(w)}{(w-z)^2}\,dw
\qquad (|z|\le 1).
\]
Taking moduli and using $\min_{|w|=R,\,|z|=1}|w-z|=R-1$ gives
\(
|p'(z)|\le \frac{R}{(R-1)^2}\,\max_{|w|=R}|p(w)|.
\)
By the maximum modulus principle, $\max_{|w|=R}|p(w)|\le R^n\max_{|u|=1}|p(u)|$.
Optimizing in $R$ yields the inequality with constant $n$ (e.g.\ let $R\downarrow 1$
and use a standard smoothing argument, or consider $f(\zeta):=\zeta p'(\zeta)/(n p(\zeta))$
which is analytic in $|\zeta|<1$ and has $|f|\le 1$ on $|\zeta|=1$).
Sharpness holds for $p(z)=c\,z^{n}$, where equality is attained.
\end{proof}

\begin{remark}[Trigonometric form]
If $T(\theta)=\sum_{k=-n}^n c_k e^{ik\theta}$ is a trigonometric polynomial of
degree $n$, then
\[
\|T'\|_{L^\infty(\mathbb T)} \le n\,\|T\|_{L^\infty(\mathbb T)}.
\]
This is the same inequality under the identification $p(e^{i\theta})=T(\theta)$.
\end{remark}

\subsection{Erdős--Lax refinement (zero-free case)}
\begin{theorem}[Erdős--Lax]
\label{thm:erdos-lax}
If $p$ has no zeros in $\{z:|z|<1\}$, then
\[
\max_{|z|=1}|p'(z)| \;\le\; \frac{n}{2}\,\max_{|z|=1}|p(z)|,
\]
and the factor $\tfrac{1}{2}$ is best possible.
\end{theorem}

\begin{proof}[Proof sketch]
When $p$ is zero-free in the disk, the function
\(
F(z):=\dfrac{z\,p'(z)}{p(z)}
\)
is analytic for $|z|<1$. Consider 
\(
G(z):=1-\frac{1}{n}F(z)=1-\frac{z\,p'(z)}{n\,p(z)},
\)
which has nonnegative real part in the unit disk (Herglotz--Carathéodory
theory for analytic functions with positive real part). Boundary estimates for
such functions imply $|F(e^{it})|\le \tfrac{n}{2}$ for a.e.\ $t$, hence the
stated inequality on $|z|=1$. Sharpness is witnessed by extremals tending to
$p(z) = c\,(z-a)^n$ with $|a|\downarrow 1$.
\end{proof}

\subsection{Markov's inequality on \([-1,1]\) and a precise contrast}
\begin{theorem}[Markov, 1889]
\label{thm:markov}
If $q$ is a real (algebraic) polynomial of degree $n$, then
\[
\max_{x\in[-1,1]}|q'(x)| \;\le\; n^2\,\max_{x\in[-1,1]}|q(x)|,
\]
and the exponent $n^2$ is sharp (attained, up to the constant, by the
Chebyshev polynomials $T_n$).
\end{theorem}

\begin{proof}[Proof sketch and extremals]
Normalize so that $\|q\|_{[-1,1]}\le 1$. Among all degree-$n$ polynomials with
unit sup-norm, $T_n(\cos\theta)=\cos(n\theta)$ oscillates between $\pm 1$ with
equal ripple and has maximal derivative at the endpoints:
$T_n'(1)=n^2$ and $\|T_n'\|_{[-1,1]}=n^2$. A standard argument (via
convexity of the feasible set and extreme point/ripple characterizations,
or via potential theory/Markov brothers' method) shows $T_n$ is extremal,
yielding the sharp constant $n^2$.
\end{proof}

\paragraph{Why $n$ on $\mathbb T$ but $n^2$ on $[-1,1]$?}
Two complementary viewpoints:
\begin{itemize}
  \item \emph{Geometry of the domain.} On the smooth closed curve \(|z|=1\),
        Cauchy’s estimate couples the growth of $p'$ linearly with degree
        (one derivative $\leftrightarrow$ one power of $n$). On \([-1,1]\),
        the boundary has endpoints where extremals concentrate curvature
        (boundary layer near $\pm 1$). This produces an extra factor of $n$.
  \item \emph{Conformal map heuristic.} The Joukowski map
        $J(\zeta)=\frac{1}{2}(\zeta+\zeta^{-1})$ sends the circle to the
        interval. Pulling a degree-$n$ algebraic polynomial on \([-1,1]\) back
        to the circle yields a \emph{sum of two} degree-$n$ terms (from
        $\zeta^k$ and $\zeta^{-k}$), and differentiation on the circle then
        costs a factor $n$ while the pullback/pushforward introduces another
        factor $n$, giving the $n^2$ scaling.
\end{itemize}

\paragraph{Quantitative refinements on \([-1,1]\).}
More precise inequalities localize the derivative size:
\[
|q'(x)| \;\le\; \frac{n^2}{\sqrt{1-x^2}}\,\|q\|_{[-1,1]},
\qquad |x|<1,
\]
and near the endpoints $x=\pm 1$ the $\sqrt{1-x^2}$ denominator captures the
boundary layer responsible for the $n^2$ growth.

\subsection{Further connections and nearby inequalities}
\begin{itemize}
  \item \textbf{$L^p$ versions on $\mathbb T$.} For $1\le p\le\infty$ and
        trigonometric polynomials $T$ of degree $n$,
        \(\|T'\|_{L^p(\mathbb T)}\le n\,\|T\|_{L^p(\mathbb T)}\)
        (Riesz--Zygmund/Nikolskii-type estimates).
  \item \textbf{Zero-free gains.} Erdős--Lax (Thm.~\ref{thm:erdos-lax})
        demonstrates how excluding zeros from the disk halves the constant
        from $n$ to $n/2$. Analogous ``stability under zero separation''
        phenomena appear in inequalities on other domains.
  \item \textbf{Intervals and orthogonal bases.} On \([-1,1]\), Chebyshev
        polynomials serve as extremals for both the value problem (minimax
        deviation) and the derivative problem (Markov). This dual role
        explains the sharpness and suggests practical preconditioners in
        spectral methods.
\end{itemize}

\subsection{Bernstein polynomials for approximation (a different ``Bernstein'')}
Distinct from the derivative inequality, Bernstein's constructive proof of the
Weierstrass theorem uses the positive linear operators
\[
(B_n f)(x)
=\sum_{k=0}^{n} f\!\left(\frac{k}{n}\right)\binom{n}{k}x^{k}(1-x)^{n-k},
\qquad x\in[0,1],
\]
which satisfy $B_n f \to f$ uniformly on $[0,1]$ for every continuous $f$.
These polynomials are linked in spirit (they control oscillation and are built
from binomial bases) but address approximation of \emph{functions} rather than
derivative bounds for a \emph{given polynomial}. The positivity and shape
preservation of $B_n$ provide quantitative moduli-of-continuity estimates:
\[
\|B_n f - f\|_{[0,1]} \;\le\; \omega\!\left(f; \sqrt{\frac{x(1-x)}{n}}\right),
\]
and, for Lipschitz $f$, a global $O(n^{-1/2})$ rate.

\subsection{Summary table}
\begin{center}
\renewcommand{\arraystretch}{1.2}
\begin{tabular}{lll}
\toprule
\textbf{Setting} & \textbf{Inequality} & \textbf{Sharp constant / extremals}\\
\midrule
Unit circle $\{|z|=1\}$ & $\displaystyle \max_{|z|=1}|p'|\le n\,\max_{|z|=1}|p|$
& $n$; $p(z)=c\,z^n$ \\
Zero-free in $|z|<1$ & $\displaystyle \max_{|z|=1}|p'|\le \tfrac{n}{2}\max_{|z|=1}|p|$
& $\tfrac{n}{2}$; limiting $(z-a)^n$, $|a|\downarrow 1$ \\
Interval $[-1,1]$ & $\displaystyle \max_{[-1,1]}|q'|\le n^2\,\max_{[-1,1]}|q|$
& $n^2$; $q=T_n$ \\
\bottomrule
\end{tabular}
\end{center}

\subsection{Drop-in theorems for use elsewhere}
\begin{theorem}[Bernstein, unit circle]
Let $p$ be a degree-$n$ polynomial. Then
$\max_{|z|=1}|p'(z)| \le n \max_{|z|=1}|p(z)|$, with equality for $p(z)=c\,z^n$.
\end{theorem}

\begin{theorem}[Erdős--Lax]
If $p$ has no zeros in $|z|<1$, then
$\max_{|z|=1}|p'(z)| \le \tfrac{n}{2}\max_{|z|=1}|p(z)|$, and $\tfrac{1}{2}$ is sharp.
\end{theorem}

\begin{theorem}[Markov]
For a real polynomial $q$ of degree $n$,
$\max_{x\in[-1,1]}|q'(x)| \le n^2 \max_{x\in[-1,1]}|q(x)|$, with sharpness at $q=T_n$.
\end{theorem}

\subsection{Practical takeaways}
\begin{itemize}
  \item On smooth closed curves (e.g.\ the circle), a single derivative costs
        a single power of $n$; excluding interior zeros improves the constant
        by a factor 2.
  \item On intervals with endpoints, boundary layers inflate the growth to
        $n^2$; Chebyshev structure captures the extremal behavior and guides
        numerical design.
  \item ``Bernstein polynomials'' for approximation are different objects:
        positive linear approximants that converge uniformly and provide
        constructive bounds for $\|B_n f - f\|$.
\end{itemize}

\section{Density of Polynomials and the Bernstein Expectation Proof}
\label{sec:poly-density-bernstein}

This section records concise answers to two related questions from our chat:
(1) \emph{When are (algebraic) polynomials dense?} and (2) \emph{How does the
Bernstein expectation argument prove uniform approximation on \([0,1]\)?}

\subsection*{Where polynomials are (and are not) dense}

\begin{itemize}
  \item \textbf{Weierstrass (univariate, compact).}
  In \(C([a,b])\) with the sup norm, algebraic polynomials are dense.

  \item \textbf{Stone--Weierstrass (multivariate, compact).}
  Let \(K\subset\mathbb{R}^n\) be compact. The algebra of polynomials in
  \(n\) variables separates points and contains the constants, hence it is
  uniformly dense in \(C(K)\).

  \item \textbf{\(L^p\) on finite measure sets.}
  For \(1\le p<\infty\), polynomials are dense in \(L^p([a,b])\) and more
  generally in weighted \(L^p([a,b],w\,dx)\) whenever \(w>0\) a.e.

  \item \textbf{Analytic function spaces (classics).}
  Polynomials are dense in the disc algebra \(A(\mathbb{D})\) (uniformly on
  \(\overline{\mathbb{D}}\)) and in Hardy spaces \(H^p\) on the unit disc
  (convergence in the \(H^p\) norm). Trigonometric polynomials are dense in
  \(C(\mathbb{T})\) and \(L^p(\mathbb{T})\).

  \item \textbf{Failure cases.}
  In \(C_b(\mathbb{R})\) with the sup norm, polynomials are not dense:
  polynomials are unbounded and cannot uniformly approximate bounded
  functions on all of \(\mathbb{R}\). In \(L^p(\mathbb{R})\) with Lebesgue
  measure, nonzero polynomials do not even belong to the space; with
  decaying weights (e.g.\ Gaussian) density can be recovered.

  \item \textbf{Müntz--Szász (sparse monomials).}
  For exponents \(\{\lambda_k\}\subset(0,\infty)\), the linear span of
  \(\{1,x^{\lambda_k}:k\ge1\}\) is dense in \(C([0,1])\) iff
  \(\sum_k \lambda_k^{-1}=\infty\).
\end{itemize}

\subsection*{Bernstein polynomials via expectations}

Fix a continuous \(f:[0,1]\to\mathbb{R}\). For \(x\in[0,1]\), let
\(S_n\sim \mathrm{Bin}(n,x)\) (i.e.\ \(S_n=\sum_{i=1}^n X_i\) with
\(X_i\stackrel{iid}{\sim}\mathrm{Bernoulli}(x)\)). The \emph{Bernstein
operator} is
\[
B_nf(x)\;=\;\mathbb{E}\!\left[f\!\left(\frac{S_n}{n}\right)\right]
=\sum_{k=0}^n f\!\left(\frac{k}{n}\right)\binom{n}{k}x^k(1-x)^{n-k},
\]
a polynomial of degree \(\le n\).

\paragraph{Uniform approximation.}
Let the modulus of continuity be
\(\omega_f(\delta)=\sup\{|f(u)-f(v)|:\,|u-v|\le\delta,\;u,v\in[0,1]\}\),
with \(\omega_f(\delta)\to0\) as \(\delta\downarrow0\). For any
\(\delta>0\),
\[
\begin{aligned}
|B_nf(x)-f(x)|
&=\big|\mathbb{E}\big[f(S_n/n)-f(x)\big]\big| \\
&\le \mathbb{E}\!\left[|f(S_n/n)-f(x)|\;\mathbf{1}_{\{|S_n/n-x|\le\delta\}}\right]
   +\mathbb{E}\!\left[|f(S_n/n)-f(x)|\;\mathbf{1}_{\{|S_n/n-x|>\delta\}}\right] \\
&\le \omega_f(\delta)+2\|f\|_\infty\;\mathbb{P}(|S_n/n-x|>\delta).
\end{aligned}
\]
Since \(\operatorname{Var}(S_n/n)=x(1-x)/n\le 1/(4n)\), Chebyshev yields
\[
\mathbb{P}(|S_n/n-x|>\delta)\le\frac{1}{4n\delta^2},
\]
so, uniformly in \(x\in[0,1]\),
\[
\|B_nf-f\|_\infty \;\le\; \omega_f(\delta) + \frac{\|f\|_\infty}{2n\delta^2}.
\]
Given \(\varepsilon>0\), choose \(\delta\) with \(\omega_f(\delta)<\varepsilon/2\),
then \(n\) so that \(\|f\|_\infty/(2n\delta^2)<\varepsilon/2\); hence
\(\|B_nf-f\|_\infty<\varepsilon\).

\paragraph{Exponential tails (optional).}
Hoeffding’s inequality improves the tail:
\[
\mathbb{P}(|S_n/n-x|>\delta)\le 2e^{-2n\delta^2},
\quad\Rightarrow\quad
\|B_nf-f\|_\infty \le \omega_f(\delta)+4\|f\|_\infty e^{-2n\delta^2}.
\]

\paragraph{Rates under smoothness (optional).}
If \(f\) is Lipschitz with constant \(L\),
\[
|B_nf(x)-f(x)|\le L\,\mathbb{E}|S_n/n-x|
\le L\sqrt{\mathrm{Var}(S_n/n)} \le \frac{L}{2\sqrt{n}}.
\]
If \(f\in C^2([0,1])\), a Taylor expansion at \(x\) and
\(\mathbb{E}(S_n/n-x)=0\), \(\mathbb{E}(S_n/n-x)^2=x(1-x)/n\) give the classical
bias:
\[
B_nf(x)-f(x) \;=\; \tfrac12 f''(x)\,\frac{x(1-x)}{n} \;+\; o\!\left(\frac{1}{n}\right).
\]

\medskip
\noindent\textbf{Summary.} The Bernstein operator \(B_n\) realizes the law of
large numbers at the level of functions: sampling at the empirical mean
\(S_n/n\) concentrates around \(x\), and the continuity of \(f\) transfers this
concentration into uniform approximation by polynomials.


\begin{thebibliography}{9}

\bibitem{BLM2013}
S. Boucheron, G. Lugosi, and P. Massart (2013).
\newblock \emph{Concentration Inequalities: A Nonasymptotic Theory of Independence}.
\newblock Oxford University Press.
\end{thebibliography}

\end{document}
